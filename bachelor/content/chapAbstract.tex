\chapter*{Kurzfassung/\emph{Abstract}}
\label{chap:abstract}

%context
Heutzutage gewinnen Large Language Models (LLMs) wie ChatGPT und Gemini immer weiter an Beliebtheit.
In Kombination mit relevanten, häufig nicht öffentlichen Dokumenten können Sie noch hilfreicher sein.

%need / have
Wenn man mithilfe von LLMs jetzt relevante Daten in einer Datenbank sucht, um damit kontextabhängige Fragen zu beantworten hat man ein Retrieval-Augmented Generation (RAG).
%need / want
Wie gut RAGs jedoch funktioniert, hängt von vielen unterschiedlichen Faktoren ab.
RAGs manuell zu bewerten ist ein zeit- und kostenintensiver Prozess.\\
%task
Retrieval Augmented Generation Assessment (RAGAS) hat ein System entwickelt, um RAGs automatisch zu bewerten.
%object of this doc
Wie gut diese automatisierte Bewertung von RAGs mithilfe von RAGAS funktioniert ist der Hauptfokus dieser Arbeit.
Dabei werden besonders drei Bereiche untersucht, die generierten Fragebögen, die Bewertung und die Zuverlässigkeit bei mehreren Wiederholungen.
%findings
Wir haben beobachtet, dass das genutzte LLM eine große Rolle spielt.
Bessere LLMs haben sowohl weniger Fehler während der Generierung der Fragebögen gemacht als auch allgemein bessere Fragen generiert.
Es hat sich außerdem gezeigt, dass Fehler aus den Fragebögen sich durch die Bewertung ziehen und dadurch die Bewertung negativ beeinflussen.
Die Metriken und Bewertungen waren konstant über mehrere Bewertungen hinweg und es waren nur minimale Schwankungen festzustellen.

%conclusion
Zusammenfassen lässt sich sagen, dass Ragas nicht als alleiniger Faktor eingesetzt werden kann und mindestens bei der Testset-Generierung eine menschliche Überprüfung stattfinden sollte.
Für die meisten Fälle ist zudem ein LLM mit vielen Parametern zu empfehlen, da dieses bessere Ergebnisse liefert.

%perspective
Zukünftig lässt sich Ragas mit weiterentwickelten LLMs und einer verbesserten Fragebogengenerierung vielleicht komplett automatisieren
!TODO!




Schlagwörter/Schlüsselwörter: gegebenenfalls Angabe von 3 bis 10 Schlagwörtern.
LLM, KI, RAG, Ragas, Automatisierung