\begin{figure}[htbp]
    \centering
    \resizebox{\textwidth}{!}{
    \begin{tikzpicture}[
        node distance=1.8cm and 1.5cm,
        model/.style={rectangle, draw, fill=pink!20, text width=3cm, text centered, minimum height=1cm, rounded corners=3pt},
        storage/.style={rectangle, draw, fill=blue!20, text width=3cm, text centered, minimum height=1cm, rounded corners=3pt},
        process/.style={rectangle, draw, fill=green!20, text width=3cm, text centered, minimum height=1cm, rounded corners=3pt},
        data/.style={rectangle, draw, fill=red!20, text width=3cm, text centered, minimum height=1cm, rounded corners=3pt},
        arrow/.style={thick,->,>=stealth},
        stage_label/.style={font=\bfseries, text width=3.5cm, text centered, node distance=0.8cm}
    ]
        % Stage 1: Datenverarbeitung
        \node[data] (doc) {Dokumentensammlung};
        \node[stage_label] (stage1_label) [above=of doc] {Phase 1: Datenverarbeitung};
        \node[process] (dl) [below=of doc] {Dokumentenlader};
        \node[data] (chunks) [below=of dl] {\textbf{Textabschnitte}};
        \node[model] (emb1) [below=of chunks, xshift=-2cm] {Einbettungsmodell für die Dokumentenverarbeitung};
        \node[storage] (vstore) [below=of chunks, xshift=2cm] {Vektorspeicher};
        \node[storage] (chroma) [below=of vstore] {ChromaDB-Sammlung};

        % Stage 2: Testset-Erstellung
        \node[process] (tsgen) [right=4.5cm of doc, yshift=-1.8cm] {Testset-Generator};
        \node[stage_label] (stage2_label) [above=of tsgen] {Phase 2: Testset-Erstellung};
        \node[model] (llm1) [above left=0.8cm and 0.4cm of tsgen] {LLM für die Testset-Generierung};
        \node[model] (emb2) [below left=0.8cm and 0.4cm of tsgen] {Einbettungsmodell für die Testset-Generierung};
        \node[data] (testset) [below=of tsgen] {Testset};

        % Stage 3: RAG-Bewertung
        \node[process] (eval) [right=4.5cm of tsgen] {Bewertung};
        \node[stage_label] (stage3_label) [above=of eval] {Phase 3: RAG-Bewertung};
        \node[model] (llm2) [above left=0.8cm and 0.4cm of eval] {\textbf{Zu bewertendes Modell}};
        \node[model] (llm3) [below left=0.8cm and 0.4cm of eval] {LLM-Modell als Richter};    
        \node[data] (report) [below=of eval] {Bewertungsbericht};
        \node[storage] (repfile) [below=of report] {Berichtsdateien};

        % Verbindungen
        % Dokumentenverarbeitungsfluss
        \draw[arrow] (doc) -- (dl);
        \draw[arrow] (dl) -- (chunks);
        \draw[arrow] (chunks) -| (emb1);
        \draw[arrow] (chunks) -| (vstore);
        \draw[arrow] (emb1) -- (vstore);
        \draw[arrow] (vstore) -- (chroma);

        % Testset-Generierungsfluss
        \draw[arrow] (doc) -- (tsgen);
        \draw[arrow] (llm1) -- (tsgen);
        \draw[arrow] (emb2) -- (tsgen);
        \draw[arrow] (tsgen) -- (testset);

        % Bewertungsfluss
        \draw[arrow] (testset.north east) to[bend left=15] (eval.west);
        \draw[arrow] (chroma.east) -| (eval.west);
        \draw[arrow] (llm2) -- (eval);
        \draw[arrow] (llm3) -- (eval);
        \draw[arrow] (eval) -- (report);
        \draw[arrow] (report) -- (repfile);
    \end{tikzpicture}
    }
    \caption{Flussdiagramm des RAG-Bewertungsprozesses, das die Interaktion zwischen verschiedenen Komponenten und Modellen zeigt. Spezifische Modellnamen (z.B. gpt-4-turbo, text-embedding-3-large) sind im Haupttext beschrieben.}
    \label{fig:rag-flow}
\end{figure}