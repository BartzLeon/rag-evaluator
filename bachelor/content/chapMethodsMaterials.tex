\chapter{Methoden und Materialien}

\subsection{Tools}

Für das RAG selbst und die Bewertung der RAGs werden gewisse Tools benötigt.
Diese Tools werden im Folgenden erklärt.

\subsubsection{Ollama}
Ollama ist ein Open-Source LLM-Server, der auf einem eigenen Computer oder in der Cloud ausgeführt wird.
Es können verschiedene Open-Source LLMs und Embedding-Modelle ausgeführt werden. In unserem Fall werden die Modelle ollama/nomic-embed-text und ollama/deepseek-r1:32b verwendet.

\subsubsection{ChromaDB}
ChromaDB ist eine Open-Source Vektordatenbank, die für die Speicherung von Vektordatenbanken verwendet wird.
Diese Datenbank wird in unseren Experimenten sowohl für die Open-Source Modelle verwendet, als auch für die Closed-Source Modelle von OpenAI.
Damit soll eine einheitliche Grundlage für die Experimente geschaffen werden.

\subsubsection{RAGAS}
RAGAS ist eine Bibliothek, die Werkzeuge bereitstellt, um die Evaluation von Large Language Model (LLM) Anwendungen zu verbessern.
Sie wurde entwickelt, um die Bewertung von LLM-Anwendungen einfach und zuverlässig zu gestalten.
\footnote{\url{https://docs.ragas.io/en/stable/#frequently-asked-questions}}

RAGAS ist ein Open-Source-Tool und liefert neben dem Tool selber hilfreiche Dokumentation für die Metriken und die Bewertung von RAGs.
Für diese Arbeit sind die Funktionen der Testset-Generierung und die damit ermöglichte Bewertung der RAGs relevant.
Es gibt noch weitere Funktionen wie die automatische Generation von Interessengruppen, diese sind für diese Arbeit jedoch nicht relevant.

Was RAGAS besonders macht zu vorherigen Tools ist, dass keine "reference answer" benötigt wird.
RAGAS ist beliebt, da es sich gut mit vielen Tools integriert.

\subsubsection{Giskard}
Giskard ist ein teils Open-Source-Tool, welches die Bewertung von RAGs unterstützt.
Der Schwerpunkt von Giskard liegt eher auf der generellen Bewertung von LLMs.
Dazu gehören unter anderem Prompt-Injectionen, Halluzinationen und andere Fehler, welche durch die Verwendung von LLMs entstehen können.



\subsection{Daten}

\subsection{Evaluation}

\subsection{Datenbank}