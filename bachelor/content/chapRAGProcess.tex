\section{RAG-Bewertungsprozess}

Das in Abbildung~\ref{fig:rag-flow} gezeigte Flussdiagramm veranschaulicht die drei Hauptphasen des RAG-Bewertungsprozesses:

\begin{enumerate}
    \item \textbf{Dokumentenverarbeitung:}
    \begin{enumerate}
        \item Dokumente werden geladen und in Abschnitte unterteilt
        \item Textabschnitte werden eingebettet
        \item Eingebettete Vektoren werden in ChromaDB gespeichert
    \end{enumerate}
    
    \item \textbf{Fragebogenerstellung:}
    \begin{enumerate}
        \item Generierung von Fragen mittels LLM
        \item Erstellung von Testsets mit Fragen und Referenzantworten
    \end{enumerate}
    
    \item \textbf{Bewertungsprozess:}
    \begin{enumerate}
        \item Verwendung des generierte Testsets
        \item Abruf des Kontexts aus der ChromaDB
        \item Bewertung der Modellantworten mittels LLM als Richter
        \item Generierung umfassender Bewertungsberichte
    \end{enumerate}
\end{enumerate}


\begin{figure}[!ht]
    \centering
    \resizebox{\textwidth}{!}{
    \begin{tikzpicture}[
        node distance=1.8cm and 1.5cm,
        model/.style={rectangle, draw, fill=pink!20, text width=3cm, text centered, minimum height=1cm, rounded corners=3pt},
        storage/.style={rectangle, draw, fill=blue!20, text width=3cm, text centered, minimum height=1cm, rounded corners=3pt},
        process/.style={rectangle, draw, fill=green!20, text width=3cm, text centered, minimum height=1cm, rounded corners=3pt},
        data/.style={rectangle, draw, fill=red!20, text width=3cm, text centered, minimum height=1cm, rounded corners=3pt},
        arrow/.style={thick,->,>=stealth},
        stage_label/.style={font=\bfseries, text width=3.5cm, text centered, node distance=0.8cm}
    ]
        % Stage 1: Datenverarbeitung
        \node[data] (doc) {Dokumenten\-sammlung};
        \node[stage_label] (stage1_label) [above=of doc] {Phase 1: Datenverarbeitung};
        \node[process] (dl) [below=of doc] {Dokumentenlader};
        \node[data] (chunks) [below=of dl] {\textbf{Textabschnitte}};
        \node[model] (emb1) [below=of chunks, xshift=-2cm] {Einbettungsmodell für die Dokumentenverarbeitung};
        \node[storage] (vstore) [below=of chunks, xshift=2cm] {Vektorspeicher};
        \node[storage] (chroma) [below=of vstore] {ChromaDB-Sammlung};

        % Stage 2: Testset-Erstellung
        \node[process] (tsgen) [right=4.5cm of dl] {Testset-Generator};
        \node[stage_label] (stage2_label) at (stage1_label -| tsgen) {Phase 2: Testset-Erstellung};
        \node[model] (llm1) [above=0.8cm of tsgen] {LLM für die Testset-Generierung};
        \node[model] (emb2) [below left=0.8cm and 0.4cm of tsgen] {Einbettungsmodell für die Testset-Generierung};
        \node[data] (testset) [below right=0.8cm and 0.4 of tsgen] {Testset};

        % Stage 3: RAG-Bewertung
        \node[process] (eval) [right=4.5cm of tsgen] {Bewerter};
        \node[stage_label] (stage3_label) at (stage1_label -| eval) {Phase 3: RAG-Bewertung};
        \node[model] (llm2) [above=0.8cm of eval, xshift=-1.7cm] {\textbf{Zu bewertendes Modell}};
        \node[model] (llm3) [above=0.8cm of eval, xshift=1.7cm] {\textbf{LLM-Modell als Richter}};
        \node[data] (report) [below=of eval] {Bewertungsbericht};
        \node[storage] (repfile) [below=of report] {Berichtsdateien};

        % Verbindungen
        % Dokumentenverarbeitungsfluss
        \draw[arrow] (doc) -- (dl);
        \draw[arrow] (dl) -- (chunks);
        \draw[arrow] (chunks) -| (emb1);
        \draw[arrow] (chunks) -| (vstore);
        \draw[arrow] (emb1) -- (vstore);
        \draw[arrow] (vstore) -- (chroma);

        % Testset-Generierungsfluss
        \draw[arrow] (doc) -- (tsgen);
        \draw[arrow] (llm1) -- (tsgen);
        \draw[arrow] (emb2) -- (tsgen);
        \draw[arrow] (tsgen) -- (testset);

        % Bewertungsfluss
        \draw[arrow] (testset) -- (eval.west);
        \draw[arrow] (chroma.east) -| (eval.west);
        \draw[arrow] (llm2) -- (eval);
        \draw[arrow] (llm3) -- (eval);
        \draw[arrow] (eval) -- (report);
        \draw[arrow] (report) -- (repfile);
    \end{tikzpicture}
    }
    \caption[Flussdiagramm des RAG-Bewertungsprozesses]{Flussdiagramm des RAG-Bewertungsprozesses, das die Interaktion zwischen verschiedenen Komponenten und Modellen zeigt. Spezifische Modellnamen (z.B. gpt-4-turbo, text-embedding-3-large) sind im Haupttext beschrieben.}
    \label{fig:rag-flow}
\end{figure}

% Farblegende für das Flussdiagramm
\begin{center}
\begin{minipage}{0.85\textwidth}
\textbf{Legende für Flussdiagrammfarben:}
\begin{itemize}
    \item \colorbox{pink!20}{\strut\hspace{1.5em}} \textbf{Modell}: (z.B. LLMs, Einbettungsmodelle)
    \item \colorbox{blue!20}{\strut\hspace{1.5em}} \textbf{Speicher}: (z.B. Vektorspeicher, ChromaDB)
    \item \colorbox{green!20}{\strut\hspace{1.5em}} \textbf{Prozess}: (z.B. Dokumentenlader, Bewertung)
    \item \colorbox{red!20}{\strut\hspace{1.5em}} \textbf{Daten}: (z.B. Dokumentensammlung, Testset, Bericht)
\end{itemize}
\end{minipage}
\end{center}

Das in Abbildung~\ref{fig:rag-flow} dargestellte Diagramm hebt hervor, wie bestimmte Komponenten, wie LLM, für verschiedene Zwecke wiederverwendet werden, während separate Einbettungsmodelle für spezifische Aufgaben beibehalten werden. Dieser modulare Ansatz ermöglicht flexible Versuche mit verschiedenen Modellen und Konfigurationen, während ein konsistentes Bewertungsframework beibehalten wird.