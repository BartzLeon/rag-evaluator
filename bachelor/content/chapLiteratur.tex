\chapter{Zitate und Literaturangaben}
\label{chap:literature}
%
Fremdes Gedankengut muss immer kenntlich gemacht werden. Vor allem muss es überprüfbar und auffindbar sein. Hierzu dient die Technik des Zitierens und Belegens.
\par
Verschiedene Fachrichtungen und Studiengänge folgen spezifischen Zitierkonventionen. Wichtige Hinweise zu gängigen Zitationssystem und -stilen finden Sie in den E-Learning-Kursen des Schreibzentrums\footnote{\href{https://ilu.th-koeln.de/goto.php?target=cat\_52109\&client\_id=thkilu}{https://ilu.th-koeln.de/goto.php?target=cat\_52109\&client\_id=thkilu}}.
%
\section{Zitieren}
%Bitte legen Sie den Zitierstil immer am Anfang fest und wechseln sie ihn nicht.
\par
Wörtlich übernommene Textpassagen werden durch Anführungszeichen unten und oben (\enquote{\ldots}) kenntlich gemacht.
\par
Wenn Ihr Zitat bereits ein Zitat enthält, müssen Sie die \enquote{doppelten Anführungszeichen} im Text durch \enquote*{einfache Anführungszeichen} ersetzen. Dazu vergleiche auch Abschnitt~\ref{sec:specialCases}.
%Verwechseln Sie diese bitte nicht mit dem Apostroph oder dem Zoll-Zeichen auf Ihrer Tastatur. Die Tastaturkürzel für korrekte Anführungszeichen und andere Sonderzeichen sind bei Typefacts aufgelistet.
%
\subsection{Quellenverweise}
Für alle Zitate muss ein Quellenverweis erstellt werden. Der Quellenverweis ist eine im Zitationsstil festgelegte Kurznotation, die auf die vollständige Literaturangabe im Literaturverzeichnis verweist. Quellenverweise können \emph{entweder} im laufenden Text (anglo-amerikanische Zitierweise bzw. Harvard-Stil) \emph{oder} über eine Fußnote am unteren Ende der Seite \emph{oder} in einer Endnote am Ende des gesamten Textes erfolgen. Hier gelten unterschiedliche fachliche Konventionen, die unbedingt beachtet werden müssen.
\par
Entscheidet man sich für Kurzverweise im Textfluss oder für Endnoten, kann der Fußnotenbereich für Kommentare und für Verweise auf Stellen im eigenen Text genutzt werden.
\par
\emph{Beachten Sie die Positionen von Hochzahl und Satzzeichen:}
\par
Wenn das wörtliche Zitat selbst mit einem Punkt endet, steht dieser vor dem beendenden Anführungszeichen. Die Hochzahl folgt dann direkt danach ohne Leerzeichen. Ein Punkt für den eigenen Satz entfällt dann.
\par
Wenn das wörtliche Zitat nicht mit einem Punkt endet gibt es zwei Fälle: Steht das Zitat mitten im eigenen Satz, folgt die Hochzahl direkt nach den Anführungszeichen. Steht das Zitat am Ende des eigenen Satzes, notiert man zuerst die Anführungszeichen, dann den eigenen Satzpunkt und erst dann die Hochzahl.
\par
Damit die Quellenverweise auch bei mehreren gleichlautenden Kurztiteln oder Jahreszahlen eines Autors eindeutig bleiben, erhalten die Jahreszahlen einen zusätzlichen kleinen Buchstaben.
%Oftmals ergibt sich dies erst gegen Ende der Arbeit. Daher kann es helfen, bei der Manuskripterstellung der Jahreszahl einen vorläufigen Kurztitel beizufügen. In der Endphase lässt sich dieser durch Suchen/Ersetzen über das Textverarbeitungsprogramm austauschen bzw. entfernen.
%
\subsection{Derselbe und Ebenda}
Textverarbeitungsprogramme machen das Verweisen über \emph{derselbe} und \emph{ebenda} heute überflüssig, da man nicht mehr gezwungen ist, die gleichen Angaben wieder und wieder abzutippen. Sollten Sie sich (zum Beispiel auf Anraten Ihres Prüfenden) dennoch für dieses Verfahren entscheiden, empfehlen wir dringend, \emph{ebenda} und \emph{derselbe} etc. \emph{erst in der redaktionellen Endphase} einzufügen, weil die Bezüge erst dann klar sind.
%
\subsection{Zitate aus zweiter Hand}
Bei Zitaten aus zweiter Hand, sog. Sekundärzitaten, übernimmt man ein Zitat eines Autors oder einer Autorin, ohne sich in der Originalquelle über den Sinnzusammenhang informiert zu haben. Zitate aus zweiter Hand sind nur zulässig, wenn die Originalquelle nicht beschaffbar ist. Bei allgemein zugänglicher wissenschaftlicher Literatur sind Zitate aus zweiter Hand unbedingt zu vermeiden. Ist es nicht möglich, ein Zitat mit dem Originaltext zu vergleichen, dann notiert man zitiert nach (es folgt die Quelle, der man das Zitat entnommen hat) oder zitiert in. Dies kommt zum Beispiel vor, wenn man wissenschaftsgeschichtliche Sachverhalte aus Lehrbüchern zitiert.
%
\subsection{Abbildungen und Tabellen zitieren}
\label{sec:refFigures}
Hier gelten die gleichen Richtlinien wie für Textzitate, d.\,h. auch hier gibt es getreue und abgeänderte Übernahmen. Beim originalgetreuen Zitat können Sie eine Abbildung z. B. in PowerPoint oder einem Zeichen-/Vektorprogramm genau nachbilden. Beim Scannen ist das Copyright zu berücksichtigen; es ist nur dann erlaubt, wenn der Autor bzw. der Verlag die~--~zumeist kostenpflichtige~--~Erlaubnis dazu erteilt hat. Der Quellenverweis ist in diesem Fall wie beim wörtlichen Zitat zu gestalten. Bei eigenen Veränderungen, muss dem Quellenverweis ein Zusatz angehängt werden (z.\,B. \emph{mit geringfügigen Veränderungen}, \emph{mit eigenen Berechnungen}, usw.).
\par
Weil Grafiken häufig übernommen werden, empfiehlt es sich, eigene Grafiken oder Bebilderungen auch als solche zu kennzeichnen.
\par
In Tabellen- und Abbildungsunterschriften wird die Quelle immer direkt unter der Abbildung angegeben und nicht in einer Fuß- oder Endnote. Ins Abbildungs- und Tabellenverzeichnis gehören die Quellenangaben allerdings nicht. Dazu siehe Abschnitt~\ref{sec:captions}.
% Da sie jedoch automatisch übernommen werden, denken Sie bitte beim Aktualisieren des Verzeichnisses daran, Quellenangaben dort händisch zu löschen.
%
%
\section{Literaturverzeichnis}
%
Nachfolgend wird die Gestaltung des obligatorischen Literaturverzeichnisses erläutert.
%
\subsection{Inhalt und Anordnung der Literaturangaben}
\label{sec:bib-content}
Im Literaturverzeichnis werden alle verwendeten Schriften in alphabetischer Reihenfolge nach Autorennamen gelistet. Das Literaturverzeichnis muss alle im Text zitierten Quellen beinhalten, aber auch keine darüber hinaus gehenden. Sind Werke nicht nur in gedruckter Form, sondern auch elektronisch publiziert, sollte die Literaturangabe der Druckfassung folgen. Es sei denn, das digitale Dokument hat eine feste Dokumentennummer (DOI). Wurde im Textteil in den Quellenverweisen statt des Titels ein Buchstabenkürzel verwendet, so ist diese Angabe auch im Literaturverzeichnis kenntlich zu machen.
\par
Mehrere Werke, die ein Autor innerhalb eines Jahres veröffentlicht hat, müssen differenziert werden. Für die Kurztitel im Text sind die Jahreszahlen daher durch angehängte Kleinbuchstaben zu unterscheiden. Die Jahreszahl in der Quellenangabe bleibt jedoch unverändert ohne angehängte Buchstaben. Die Reihenfolge von a, b, c etc. richtet sich nach der Reihenfolge der Quellenverweise. Ob Vornamen abgekürzt oder ausgeschrieben werden, hängt von den Konventionen Ihres Faches ab. Bleiben Sie jedoch einheitlich.
%\par
%Die vollständige Literaturangabe wird immer mit einem Punkt abgeschlossen. Es werden nie zwei Punkte oder Komma und Punkt hintereinander gesetzt.
%
\subsection{Bibliographische Angaben im Literaturverzeichnis}
Die Reihenfolge in der Notation der Literaturangabe hängt vom Dokumententyp und von fachlichen Konventionen ab. Bei Monografien sieht sie also anders aus als bei Zeitschriftenaufsätzen, in der Chemie wieder anders als in den Geisteswissenschaften oder der Mathematik. Auch für die \emph{Interpunktion} gibt es keine einheitlichen Vorgaben.
\par
Die Daten einer Literaturangabe entnehmen Sie bei Büchern dem so genannten Titelblatt. Dies ist nicht der Buchdeckel, sondern ein bedrucktes Blatt am Buchbeginn mit den wichtigsten werk- und buchidentifizierenden Angaben. Zunächst werden Verfasser oder Herausgeber zusammen mit dem Titel genannt, darauf folgen Erscheinungs- und Druckvermerke wie Verlag, Ort und Erscheinungsjahr. Es gibt zwar eine DIN-Regel für die Titelbeschriftung, die aber sehr unterschiedlich ausgelegt wird.
\par
In den E-Learning-Kursen des Schreibzentrums\footnote{\href{https://ilu.th-koeln.de/goto.php?target=cat\_52109\&client\_id=thkilu}{https://ilu.th-koeln.de/goto.php?target=cat\_52109\&client\_id=thkilu}} finden Sie weitere Informationen zu den bibliographischen Angaben für verschiedene Publikationstypen. Die hier folgende Literaturliste ist nur ein Beispiel für ein mögliches Format.